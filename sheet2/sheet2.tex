\documentclass{article}
\usepackage{packages}

\title{\textbf{Fundamentals of Simulation Methods} \\ \vspace{5pt} \large University of Heidelberg, WiSe 2021/2022 \\ \vspace{5pt} Mandatory assignements - Set 2}
\date{\textbf{Date:} \today}
\author{\textbf{Authors:} Matteo Zortea, Anja Heim}

\begin{document}
\maketitle

\section*{Exercise 3}

\subsubsection*{a)}
By defining
\begin{equation}
    q_i \equiv \frac{\partial \mathcal{L}}{\dot q_i} \qquad i=1,2
    \label{eq:momenta_definition}
\end{equation}
one can rewrite the Lagrangian equations of motion as
\begin{gather*}
    \frac{dq_1}{dt} = \frac{\partial \mathcal{L}}{\partial q_1} = -m_2l_1l_2 \dot \phi_1 \dot \phi_2 \sin(\phi_1 - \phi_2) - (m_1 + m_2)gl_1\sin\phi_1 \\
    \frac{dq_2}{dt} = \frac{\partial \mathcal{L}}{\partial q_2} = m_2l_1l_2 \dot \phi_1 \dot \phi_2 \sin(\phi_1 - \phi_2) - m_2gl_2 \sin \phi_2
\end{gather*}
These equations determine the evolution of the momenta $q_1, q_2$. 

\subsubsection*{b)
}If we define a state vector as $y = (\phi_1, \phi_2, q_1, q_2)T$, the evolution of our system is completely 
determined by the set of equations 
\begin{equation*}
    \frac{dy}{dt} = f(y)
\end{equation*}
which, in matrix form, reads
\begin{equation}  
    \begin{pmatrix}
        \frac{d\phi_1}{dt} \\
        \\
        \frac{d\phi_2}{dt} \\
        \\
        \frac{dq_1}{dt} \\
        \\
        \frac{dq_2}{dt}
    \end{pmatrix}
    =
    \begin{pmatrix}
        \dot \phi_1 \\
        \\
        \dot \phi_2 \\
        \\
        -m_2l_1l_2 \dot \phi_1 \dot \phi_2 \sin(\phi_1 - \phi_2) - (m_1 + m_2)gl_1\sin\phi_1 \\
        \\
        m_2l_1l_2 \dot \phi_1 \dot \phi_2 \sin(\phi_1 - \phi_2) - m_2gl_2 \sin \phi_2
    \end{pmatrix}
    \label{eq:diff_eq_matrix_form}
\end{equation}
We now need to obtain the expression for $\dot \phi_i \quad i=1,2$. In order to do this, let us explicitly write the expression for $q_1, q_2$ using their definition \ref{eq:momenta_definition}.
\begin{gather*}
    q_1 = \frac{\partial \mathcal{L}}{\partial \dot q_1} = (m_1 + m_2)l_1^2 \dot \phi_1 + m_2l_1l_2 \dot \phi_2 cos(\phi_1 - \phi_2) \equiv \alpha \dot \phi_1 + \beta \dot \phi_2 \\
    q_2 = \frac{\partial \mathcal{L}}{\partial \dot q_2} = m_2 l_1 l_2 \dot \phi_1 \cos(\phi_1 - \phi_2) + m_2 l_2^2 \dot \phi_2 \equiv \beta \dot \phi_1 + \gamma \dot \phi_2
\end{gather*}
which can be written in matrix form as 
\begin{equation*}
    \begin{pmatrix}
        q_1 \\
        q_2
    \end{pmatrix}
    =
    \begin{pmatrix}
        \alpha & \beta \\
        \beta & \gamma
    \end{pmatrix}
    \begin{pmatrix}
        \dot \phi_1 \\
        \dot \phi_2
    \end{pmatrix}
\end{equation*}
We can invert these expressions using the Kramer's rule
\begin{gather*}
    \dot \phi_1 = \frac{\begin{vmatrix}q_1 & \beta \\ q_2 & \gamma \end{vmatrix}}{\begin{vmatrix}\alpha & \beta \\ \beta & \gamma \end{vmatrix}} = \frac{\gamma}{\alpha\gamma - \beta^2}q_1 - \frac{\beta}{\alpha\gamma - \beta^2}q_2 \qquad \qquad
    \dot \phi_2 = \frac{\begin{vmatrix}\alpha & q_1 \\ \beta & q_2 \end{vmatrix}}{\begin{vmatrix}\alpha & \beta \\ \beta & \gamma \end{vmatrix}} = \frac{\alpha}{\alpha \gamma - \beta^2}q_2 - \frac{\beta}{\alpha\gamma - \beta^2}q_2 
\end{gather*}
and by popping in the expressions for $\alpha, \beta, \gamma$
\begin{gather*}
    \dot \phi_1 = \frac{1}{m_1 l_1^2(1 + m_2/m_1 \sin^2(\phi_1 - \phi_2))} \, q_1 - \frac{cos(\phi_1 - \phi_2)}{m_1l_1l_2(1+m_2/m_1\sin^2(\phi_1 - \phi_2))} \, q_2 \\
    \dot \phi_2 = - \frac{1}{\mu l_2^2(1 + m_2/m_1\sin^2(\phi_1 - \phi_2))} \, q_2 - \frac{\cos(\phi_1 - \phi_2)}{m_1 l_1 l_2 (1+m_2/m_1 \sin^2(\phi_1 - \phi_2))} \, q_1
\end{gather*}
where $\mu = \frac{m_1 m_2}{m_1 + m_2}$ is the reduced mass. \\
We have now obtained all the expressions needed to solve equation \ref{eq:diff_eq_matrix_form} which now explicitly reads
\begin{equation*}
    \begin{pmatrix}
        \frac{d\phi_1}{dt} \\
        \\
        \frac{d\phi_2}{dt} \\
        \\
        \frac{dq_1}{dt} \\
        \\
        \frac{dq_2}{dt}
    \end{pmatrix}
    =
    \begin{pmatrix}
        \frac{1}{m_1 l_1^2(1 + m_2/m_1 \sin^2(\phi_1 - \phi_2))} \, q_1 - \frac{cos(\phi_1 - \phi_2)}{m_1l_1l_2(1+m_2/m_1\sin^2(\phi_1 - \phi_2))} \, q_2 \\
        \\
        -\frac{1}{\mu l_2^2(1 + m_2/m_1\sin^2(\phi_1 - \phi_2))} \, q_2 - \frac{\cos(\phi_1 - \phi_2)}{m_1 l_1 l_2 (1+m_2/m_1 \sin^2(\phi_1 - \phi_2))} \, q_1 \\
        \\
        -m_2l_1l_2 \dot \phi_1 \dot \phi_2 \sin(\phi_1 - \phi_2) - (m_1 + m_2)gl_1\sin\phi_1 \\
        \\
        m_2l_1l_2 \dot \phi_1 \dot \phi_2 \sin(\phi_1 - \phi_2) - m_2gl_2 \sin \phi_2
    \end{pmatrix}
\end{equation*}
\subsubsection*{c)}
\end{document}